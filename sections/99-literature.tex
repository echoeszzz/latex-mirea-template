\BeforeBeginEnvironment{thebibliography}{\clearpage}
\begin{thebibliography}{99\kern\bibindent}
    \bibitem{} Антамошкин, О. А. Программная инженерия. Теория и практика.
    Учебник. M: НИЦ Инфра-М, 2012.~--- 368~с.
    \bibitem{} С.А.Орлов. Программная инженерия.
    Учебник для вузов. 5-е издание обновленное и дополненное.М:
    Издательский дом "<Питер">, 2017. 812~с.
    \bibitem{} Сомон., П. И. Волшебство Kotlin : руководство / П. И. Сомон. ;
    перевод с английского А. Н. Киселева.. — Москва :
    ДМК Пресс, 2020. — 536 с. — ISBN 978-5-97060-801-2. — Текст :
    электронный // Лань : электронно-библиотечная система. — URL:
    \url{https://e.lanbook.com/book/140599} (дата обращения: 15.02.2024).
    — Режим доступа: для авториз. пользователей.
    \bibitem{} Калгина, И. С. Разработка мобильных приложений :
    учебное пособие / И. С. Калгина. — Чита : ЗабГУ, 2022. — 163 с.
    — ISBN 978-5-9293-3137-4. — Текст : электронный // Лань :
    электронно-библиотечная система.
    — URL: \url{https://e.lanbook.com/book/363323}
    (дата обращения: 15.02.2024). — Режим доступа:
    для авториз. пользователей.
    \bibitem{} 3. Алпатов, А. Н. Архитектура, проектирование
    и разработка программных средств :
    учебное пособие / А. Н. Алпатов, И. Е. Рогов. — Москва :
    РТУ МИРЭА, 2023. — 120 с. — ISBN 978-5-7339-1972-0. — Текст :
    электронный // Лань : электронно-библиотечная система.
    — URL: \url{https://e.lanbook.com/book/386189}
    (дата обращения: 15.02.2024). — Режим доступа: для авториз.
    пользователей.
\end{thebibliography}
\addcontentsline{toc}{section}{СПИСОК ИСПОЛЬЗОВАННЫХ ИСТОЧНИКОВ}